\documentclass{article}

\usepackage{amsmath}
\usepackage[T1]{fontenc}
\usepackage[margin=2.5cm]{geometry}
\usepackage{comment}
\usepackage{graphicx}
\usepackage{float}


\begin{document}
\section{Introduction}
\subsection{The problem of aerodynamic drag}
Dlaczego badamy oraz wzór na opór aerodynamiczny. Że bazujemy na modern exterior balistics i jakieś inne z literatury bo to łądnie brzmi.

\begin{comment}
\subsection{Methodology of the present work}
For simulations we choose two programs to compare the results. The first one is 
Solidworks Flow Simulation, in which we also prepared models for simulations. 
The second one is Ansys Fluent.\\\\
First we prepared the models in Solidworks and from there we exported them to 
.step (214) file format to import them to Ansys. In Ansys we used Fluent with
Meshing to prepare the mesh and then we run the simulations. Solidworks Flow Simulation
was also used to prepare the mesh and run the simulations, which we later compared with
Ansys Fluent results.\\\\
All models were tested using Parametric studies/sets for 9 different velocities 
from 0.1 to 1.0. The resulting graphs of drag coefficient vs mach number were 
compared and analyzed.
\end{comment}

\subsection{Methodology of the present work}
For simulations, two programs were chosen to compare the results. The first program, 
Solidworks Flow Simulation, was used for both CFDs and model preparation. The second 
program utilized was Ansys Fluent.\\\\
Initially, the models were prepared in Solidworks and subsequently exported to 
.step (214) file format for importation into Ansys. Within Ansys, Fluent with 
Meshing was used to prepare the mesh, followed by the execution of simulations. 
Solidworks Flow Simulation was also employed for mesh preparation and simulation execution, 
enabling subsequent comparison with results obtained from Ansys Fluent.\\\\
Parametric studies/sets were conducted for all models, encompassing nine different velocities
ranging from 0.1 to 1.0. Subsequently, resulting graphs depicting the drag coefficient 
versus Mach number were analyzed and compared.



\subsection{Tested models}
R6-Endcone, R6-No-Endcone, PrawieR5\\\\
For each set of simulations, computational domain mesh setting and graph of velocity and 
pressure for 0.6 mach will be shown.\\\\

\section{Initial study}
Work was started with the remodeled R5 model. The model was prepared in Solidworks and had 
endcone, which was the change compared to the original R5 model. The model was tested in
Solidworks Flow Simulation. The results can be seen here:

\begin{figure}[H]
    \centering
    \includegraphics[width=\textwidth]{../data/PrawieR5-Solid/PrawieR5-TR-Velocity-Mach06.png}
    \caption{Velocity graph for PrawieR5 model at Mach 0.6}
\end{figure}

\begin{figure}[H]
    \centering
    \includegraphics[width=\textwidth]{../data/PrawieR5-Solid/CD-PrawieR5.pdf}
    \caption{CD graph for PrawieR5 model at Mach 0.6}
\end{figure}
This model was only used to compare the results of older model with the new one. 

\section{R6 Endcone}
\subsection{Solidworks}
\begin{itemize}
    \item Domena i mesh
    \item Kolorki dla 0.2, 0.5, 0.8
\end{itemize}
\subsection{Ansys Fluent with meshing}
\begin{itemize}
    \item Domena i mesh
    \item Kolorki dla 0.2, 0.5, 0.8
\end{itemize}
Wykresy obu na koniec zestawić.

\section{R6 No Endcone}
\subsection{Solidworks}
\begin{itemize}
    \item Domena i mesh
    \item Kolorki dla 0.2, 0.5, 0.8
\end{itemize}
\subsection{Ansys Fluent with meshing}
\begin{itemize}
    \item Domena i mesh
    \item Kolorki dla 0.2, 0.5, 0.8
\end{itemize}
Wykresy zestawić.

\section{Results and discussion}
\begin{itemize}
    \item Wykresy CD solida
    \item Wykres CD fluenta
    \item porównanie CD dla wyników solida i ansysa jakąś tam statystyką z użyciem pythona(ja to zrobie)
    \item Podsumowanie że wyszedł lepszy dla endcone(co się zgadza z literaturą i przewidywaniami) 
    oraz jakieś tam gadu gadu o Solidzie że gorszy. 
\end{itemize}

\end{document}